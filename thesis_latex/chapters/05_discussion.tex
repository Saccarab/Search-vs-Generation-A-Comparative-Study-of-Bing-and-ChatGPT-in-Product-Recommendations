\chapter{Discussion}
\label{chap:discussion}

The findings of this study have significant implications for the future of information retrieval, digital marketing, and the architecture of the web itself.

\section{The Convergence of SEO and GEO}
The data supports the hypothesis that Generative Engine Optimization (GEO) is not a replacement for SEO, but its "final form." To be cited by an LLM, a page must first be discoverable (SEO) and then be "extractable" (GEO). The "High-Signal Mandate" suggests that as search costs fall and inference costs remain high, LLMs will increasingly rely on external retrieval, favoring content that is structured, factual, and easily parsed.

\section{The Economic Moat of Retrieval}
We conclude that the future of AI lies not in larger models, but in smarter orchestrators. By using the web as a "distributed memory layer," AI providers can reduce hallucinations and computational costs. However, this creates an economic tension: if AI agents extract answers without sending traffic, the incentive to create high-quality content on the open web may diminish.

\section{The "Invisible Web" for Humans}
Our discovery of the "Visibility Gap" implies that AI agents effectively have access to a "superior" version of the web—one that is not cluttered by ads, pagination loops, or UI suppression. This raises questions about information equity: are human users being served a degraded search experience compared to their AI counterparts?

\section{Implications for Digital Strategy}
For content creators, the message is clear: the era of optimizing for "clicks" is ending. The new era requires optimizing for "citations." This means prioritizing information density, schema markup, and neutral authority over clickbait headlines and aggressive monetization tactics.
