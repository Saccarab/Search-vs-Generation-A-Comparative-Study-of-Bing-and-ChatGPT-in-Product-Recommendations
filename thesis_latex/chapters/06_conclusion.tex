\chapter{Conclusion}
\label{chap:conclusion}

This thesis set out to quantify the divergence between traditional search engines and generative AI in the context of commercial product recommendations. By analyzing over 240 search runs and thousands of citations, we have mapped the contours of the "AI Search Filter."

\section{Summary of Findings}
We found that Generative AI does not merely mirror the top search results. Instead, it applies a rigorous, albeit opaque, quality filter that rejects a significant portion of high-ranking content. This filter prioritizes information density, structural clarity, and factual neutrality, often bypassing the "salesy" content that dominates the top of traditional SERPs.

\section{Limitations}
This study was limited to the SaaS and AI software domains and focused primarily on US-centric results. The "Deep Hunt" was capped at Rank 150, and the "Extraction Efficiency" analysis relied on automated scraping which may have missed some dynamic content.

\section{Future Work}
Future research should expand this methodology to other verticals (e.g., health, finance), explore longitudinal trends to see if the "Search Filter" evolves over time, and investigate the impact of multi-modal content (images, video) on grounding behavior.

\section{Final Thoughts}
As we transition from the age of Search to the age of Grounding, the web is being reshaped. The "10 blue links" are fading, replaced by synthesized answers. Understanding the mechanics of this new information economy is no longer optional—it is essential for anyone who wishes to be found in the age of AI.
