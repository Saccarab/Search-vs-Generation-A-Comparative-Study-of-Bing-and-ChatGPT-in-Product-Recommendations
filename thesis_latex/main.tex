\RequirePackage[l2tabu,orthodox]{nag}

% One-sided layout for digital submission
\documentclass[headsepline,footsepline,footinclude=false,oneside,fontsize=11pt,paper=a4,listof=totoc,bibliography=totoc,DIV=12]{scrbook}

% Basic packages
\usepackage[utf8]{inputenc}
\usepackage[T1]{fontenc}
\usepackage[english]{babel}
\usepackage{graphicx}
\usepackage{microtype}
\usepackage{csquotes}
\usepackage[backend=biber,style=numeric]{biblatex}
\usepackage{hyperref}
\usepackage{booktabs}
\usepackage{listings}
\usepackage{color}
\usepackage[acronym,toc]{glossaries}

% Thesis Information
\newcommand*{\getUniversity}{Technische Universität München}
\newcommand*{\getFaculty}{Department of Informatics}
\newcommand*{\getTitle}{From Ranking to Grounding: A Comparative Analysis of Information Retrieval in Search Engines and Generative AI}
\newcommand*{\getAuthor}{Ali San Kaya}
\newcommand*{\getDoctype}{Master's Thesis in Information Systems}
\newcommand*{\getSupervisor}{Supervisor Name}
\newcommand*{\getAdvisor}{Advisor Name}
\newcommand*{\getSubmissionDate}{Submission Date}
\newcommand*{\getSubmissionLocation}{Munich}

\addbibresource{bibliography.bib}
\makeglossaries

\begin{document}

\selectlanguage{english}
\pagenumbering{alph}

% Cover Page
\begin{titlepage}
	\centering
	\includegraphics[width=4cm]{figures/tum_logo.png} \\ % Placeholder for logo
	\vspace{1cm}
	{\scshape\LARGE \getUniversity \par}
	{\scshape\Large \getFaculty \par}
	\vspace{1.5cm}
	{\huge\bfseries \getTitle \par}
	\vspace{2cm}
	{\Large\itshape \getDoctype \par}
	\vspace{0.5cm}
	{\Large \getAuthor \par}
	\vfill
	supervised by\par
	\getSupervisor \par
	\vspace{0.5cm}
	advised by\par
	\getAdvisor \par
	\vfill
	{\large \getSubmissionLocation, \getSubmissionDate \par}
\end{titlepage}

\frontmatter

\chapter*{Abstract}
This thesis investigates the paradigm shift from information retrieval based on ranking (Search Engines) to information retrieval based on grounding (Generative AI). By analyzing 240 runs across 80 commercial product recommendation queries, we quantify the "Visibility Gap" between the content surfaced by Bing/Google and the content cited by ChatGPT/Gemini. Our findings reveal a significant "AI Search Filter," where generative models systematically reject high-ranking but "salesy" content in favor of information-dense sources, often bypassing the top search results entirely. This study provides empirical evidence for the transition from Search Engine Optimization (SEO) to Generative Engine Optimization (GEO).

\tableofcontents
\listoffigures
\listoftables

\mainmatter

\chapter{Introduction}
\label{chap:introduction}

The paradigm of information retrieval is undergoing a fundamental shift. For two decades, the dominant model of web search has been \textit{ranking}: a process where algorithms index billions of documents and present a ranked list of blue links to the user, optimizing for click-through rates and relevance. However, the emergence of Large Language Models (LLMs) has introduced a competing paradigm: \textit{grounding}. In this new model, artificial intelligence acts as an orchestrator, retrieving information from the web not to present links, but to synthesize a direct answer. This transition from "Search Engine Optimization" (SEO) to "Generative Engine Optimization" (GEO) represents not merely a change in user interface, but a structural transformation of the internet's information economy.

\section{Motivation}
The commercial implications of this shift are profound, particularly in high-stakes domains such as product discovery. Recent industry reports indicate that commercial queries trigger a web search in over 53\% of ChatGPT conversations \cite{profound2026}. Unlike traditional search engines, which drive traffic to external websites, generative AI engines consume content to generate self-contained responses. This "zero-click" phenomenon raises critical questions about the visibility of information. If an LLM synthesizes a product recommendation, which sources does it cite? Does it prefer the same high-authority domains that rank at the top of Google and Bing, or does it bypass the "human web" to access a deeper, "agentic web"?

\section{Problem Statement}
Despite the rapid adoption of Retrieval-Augmented Generation (RAG) systems, the mechanisms by which these models select, filter, and synthesize information remain opaque. Preliminary observations suggest a significant divergence between the content visible to human users on Search Engine Result Pages (SERPs) and the content retrieved by AI agents. This thesis defines this divergence as the "AI Search Filter"—a measurable gap between the ranking priorities of traditional search algorithms and the grounding priorities of generative models.

\section{Research Questions}
To quantify this divergence, this thesis investigates the following core research questions:

\begin{itemize}
    \item \textbf{RQ1 (The Visibility Gap):} To what extent do the sources cited by Generative AI (ChatGPT, Gemini) overlap with the top-ranked results in traditional search engines (Bing, Google)?
    \item \textbf{RQ2 (The Search Filter):} What are the characteristics of the "rejected" content—pages that rank highly in SERPs but are ignored by Generative AI?
    \item \textbf{RQ3 (Extraction Efficiency):} How does the "Content DNA" of a webpage (e.g., structure, tone, schema) influence its likelihood of being cited and synthesized by an LLM?
    \item \textbf{RQ4 (Stochasticity):} How consistent are AI-generated product recommendations across multiple independent runs of the same query?
\end{itemize}

\section{Thesis Structure}
This thesis is organized as follows: Chapter \ref{chap:methodology} details the comparative methodology used to analyze 240 runs across 80 product recommendation prompts. Chapter \ref{chap:results} presents the empirical findings regarding citation overlap and the "Search Filter" effect. Chapter \ref{chap:discussion} discusses the theoretical implications of the shift from SEO to GEO, and Chapter \ref{chap:conclusion} offers concluding remarks and future research directions.

\chapter{Theoretical Background}
\label{chap:background}

This chapter establishes the theoretical framework for understanding the transition from traditional search engines to generative answer engines. It traces the evolution of information retrieval, defines the mechanics of Retrieval-Augmented Generation (RAG), and explores the economic and technical drivers behind the shift from SEO to GEO.

\section{The Evolution of Search Optimization}
For the past two decades, the information economy has been governed by the principles of Search Engine Optimization (SEO). In this paradigm, content creators optimized their digital assets to rank highly on Search Engine Result Pages (SERPs), primarily targeting a "10 blue links" interface.

\subsection{Traditional SEO}
Traditional SEO focuses on keyword density, backlink authority, and technical performance (e.g., load speeds, mobile responsiveness). The objective is visibility within a ranked list, where the user is expected to click through to the source to consume information.

\subsection{The Emergence of AEO and GEO}
With the rise of voice assistants and featured snippets, a new discipline emerged: Answer Engine Optimization (AEO), which focused on providing the "single best answer" for zero-click searches. This has now evolved into Generative Engine Optimization (GEO), where the goal is not merely to be ranked, but to be \textit{cited} and \textit{synthesized} by Large Language Models (LLMs) within conversational interfaces.

\section{Retrieval-Augmented Generation (RAG)}
Retrieval-Augmented Generation (RAG) represents the architectural backbone of modern AI search. It was developed to address the limitations of early LLMs, which relied purely on "parametric knowledge"—static information frozen at the time of training.

\subsection{The "Stochastic Parrot" Problem}
Early models were prone to "hallucinations" and could not access real-time data, leading to the "stale data" bottleneck. RAG decouples the reasoning capabilities of the LLM from its knowledge base, allowing the model to act as a reasoning engine rather than a database.

\subsection{The RAG Workflow}
The standard RAG workflow consists of three stages:
\begin{enumerate}
    \item \textbf{Retrieval:} The model identifies an information need and generates a search query to retrieve relevant documents (grounding chunks).
    \item \textbf{Augmentation:} These documents are injected into the model's context window.
    \item \textbf{Generation:} The model synthesizes a response based on the retrieved context, citing its sources to ensure factuality.
\end{enumerate}

\section{The Shift from Search to Grounding}
A pivotal moment in this transition occurred on August 11, 2025, when Microsoft officially decommissioned its legacy Bing Search APIs in favor of "Grounding with Bing Search" as part of the Azure AI Agents ecosystem. This marked a formal recognition that the "Human Web" (optimized for ranking and engagement) is distinct from the "Agent Web" (optimized for grounding and information extraction).

\subsection{Retrieval Asymmetry}
This thesis posits that a "Retrieval Asymmetry" has emerged. Traditional search engines prioritize content based on human engagement signals, whereas grounding engines prioritize content based on information density and extraction potential. This divergence creates the "Search Filter" investigated in this study.

\section{The Economic and Technical Necessity of Retrieval}
The shift to RAG is driven not just by accuracy, but by fundamental economic constraints known as the "Compute Wall."

\subsection{Inference vs. Retrieval Costs}
Inference (the process of an LLM "thinking" or generating tokens) is exponentially more expensive than Retrieval (traditional database or index lookups). It is computationally infeasible to train a model frequently enough to keep up with the real-time web. Therefore, Retrieval becomes the only scalable solution for accessing dynamic data such as pricing, news, and product availability.

\subsection{Sam Altman's "Tiny Model" Vision}
This architectural shift aligns with the vision articulated by industry leaders. As Sam Altman noted, the ideal AI architecture may not be a massive model containing all knowledge, but a "very tiny model with superhuman reasoning" that relies on external tools to "think, search, simulate, and solve." In this framework, the web transforms from a destination for humans into a \textbf{distributed memory layer} for AI orchestrators.

\chapter{Methodology}
\label{chap:methodology}

This study employs a comparative analysis framework to measure the divergence between traditional information retrieval (Search) and generative information retrieval (GenAI). The methodology is designed to isolate the "grounding" behavior of LLMs by controlling for geographical context, query intent, and temporal variance.

\section{Data Collection Pipeline}
The dataset consists of 80 unique product recommendation queries focused on the SaaS and AI software domains (e.g., "Best AI video translators", "Top-rated transcription software"). These queries were selected based on high commercial intent and real-world usage patterns.

To ensure statistical significance and capture the stochastic nature of RAG systems, each prompt was executed in three independent runs, resulting in a total of 240 data points. The data collection pipeline integrated four distinct sources:

\begin{enumerate}
    \item \textbf{ChatGPT Data:} Full response generation including inline citations, "searched for" queries, and recommended entities.
    \item \textbf{Bing Search Data:} Retrieval of the Top 30 organic results for each query, plus a "Deep Hunt" retrieval of results up to Rank 150 to detect buried citations.
    \item \textbf{Gemini Data:} Extraction of \texttt{groundingMetadata}, specifically distinguishing between \texttt{groundingChunks} (the retrieved consideration set) and \texttt{groundingSupports} (the actually cited set).
    \item \textbf{Google SERP Data:} Retrieval of the Top 20 organic results via SerpApi to serve as a control group for the Gemini analysis.
\end{enumerate}

\section{The "Deep Hunt" Protocol}
A critical methodological innovation of this study is the "Deep Hunt" protocol. Initial pilot runs revealed that approximately 35\% of citations generated by ChatGPT were absent from the standard Top 30 Bing results. To determine whether these citations were truly "invisible" (absent from the index) or merely "buried" (ranking poorly), the retrieval depth was expanded to Rank 150. This revealed a phenomenon of "UI Suppression," where relevant content accessible to the API-based agent was effectively invisible to human users due to pagination limits and UI clutter.

\section{Measuring the "AI Search Filter"}
For the Google Gemini analysis, a specific metric was developed to quantify the model's filtering logic. By comparing the \texttt{groundingChunks} (content retrieved by the model) against the \texttt{groundingSupports} (content used by the model) and the external Google SERP (content ranked by the search engine), we calculate the \textbf{Rejection Rate}. This metric serves as a proxy for the model's internal quality filter, allowing us to analyze why high-ranking search results fail to achieve grounding status.

\section{Content Ingestion and Enrichment}
To analyze the "Content DNA" of cited versus rejected sources, a headless browser automation pipeline was developed to fetch the full HTML content of over 3,000 unique URLs. These documents were processed through a "Unified Judge" LLM pipeline to extract metadata such as:
\begin{itemize}
    \item \textbf{Tone:} (e.g., Promotional, Informational, Salesy)
    \item \textbf{Structure:} (Presence of tables, lists, schema markup)
    \item \textbf{Authority Signals:} (Clear authorship, citation density)
\end{itemize}
This enrichment step allows for a multivariate analysis of the factors driving "Extraction Efficiency" in Generative Engine Optimization.

\chapter{Results}
\label{chap:results}

This chapter presents the empirical findings of the comparative analysis between ChatGPT, Gemini, and traditional search engines. The results are derived from 240 independent runs across 80 commercial product recommendation queries.

\section{Citation Overlap Analysis}
The primary metric for measuring the divergence between Search and GenAI is the Citation Overlap Rate.

\subsection{The Visibility Gap}
Our analysis reveals a significant "Visibility Gap." Approximately 35\% of the sources cited by ChatGPT were not found in the top 30 results of Bing. Even when the retrieval depth was expanded to Rank 150 (the "Deep Hunt"), a substantial portion of citations remained "truly invisible," suggesting that Generative AI models access a different index or prioritize content differently than the public-facing SERP.

\begin{table}[h]
    \centering
    \begin{tabular}{lc}
        \toprule
        \textbf{Metric} & \textbf{Value} \\
        \midrule
        Strict URL Match (Top 30 + Deep Hunt) & 64.93\% \\
        Domain-Only Match & 79.20\% \\
        Truly Invisible Citations & $\sim$35\% \\
        \bottomrule
    \end{tabular}
    \caption{Citation Overlap Statistics between ChatGPT and Bing}
    \label{tab:overlap_stats}
\end{table}

\subsection{The "Invisible Section" Finding}
A key discovery of the "Deep Hunt" was the identification of citations that exist in the Bing index but are effectively suppressed by the user interface.
\begin{itemize}
    \item \textbf{Page 1 Instability:} Bing's results page often fluctuates, showing between 4 and 10 results, sometimes with "infinite scroll" mechanics that break traditional pagination.
    \item \textbf{The "Page 2 Cliff":} Relevant results ranking between 11 and 15 often vanish entirely when a user navigates to the next page.
    \item \textbf{Pagination Loops:} We observed instances where requesting subsequent pages (e.g., `\&first=5`) returned the same Top 10 results, creating a loop that prevents access to deeper content.
\end{itemize}
Despite these UI barriers, ChatGPT successfully retrieved and cited links found at Rank 11-30 (the "Hidden Page 1" zone), proving that its API access bypasses the limitations imposed on human users.

\section{The "AI Search Filter"}
For Google Gemini, the analysis focused on the "Rejection Rate"—the percentage of retrieved "grounding chunks" that were discarded in the final response.

\subsection{The 0\% Rejection Anomaly}
Contrary to expectations, Gemini displayed a near 0\% rejection rate between its \texttt{groundingChunks} and \texttt{groundingSupports}. This indicates that the \texttt{groundingChunks} exposed by the API are not the raw search results, but a pre-filtered "shortlist." The true filtering occurs upstream, between the Google SERP and the \texttt{groundingChunks}.

\section{Analysis of ChatGPT Network Metadata}
A significant methodological breakthrough in this study was the extraction of internal network responses from the ChatGPT conversational interface. This metadata provides a "behind-the-scenes" look at the retrieval process that is not visible in the standard UI.

\subsection{Hidden Queries and Result Groups}
The network data reveals that for a single user prompt, ChatGPT generates multiple "hidden" search queries. For example, the prompt \textit{"Which free AI would you recommend for translating my video?"} triggered internal searches for both \textit{"free AI tools for translating videos"} and \textit{"free AI video translation services"}. 

Furthermore, the metadata includes a \texttt{search\_result\_groups\_json} field, which categorizes retrieved results by domain and provides snippets and attribution data before the final response is synthesized. This structure is remarkably similar to Gemini's \texttt{groundingMetadata}, suggesting a convergence in how major LLM providers handle web-grounding.

\subsection{Cited vs. Additional Sources}
The metadata explicitly distinguishes between \texttt{sources\_cited} (used inline) and \texttt{sources\_additional} (retrieved but only suggested).
\begin{itemize}
    \item \textbf{Citation Density:} In the analyzed run for P001, ChatGPT retrieved 15 unique domains but only cited 9 inline.
    \item \textbf{The "Additional" Filter:} The 6 domains relegated to "Additional Sources" (e.g., \texttt{usefulai.com}, \texttt{unite.ai}) were primarily listicles or directories. This suggests an internal ranking mechanism that prioritizes direct tool providers (e.g., \texttt{aivideotranslator.ai}) for inline citations while keeping secondary aggregators as background context.
\end{itemize}

\section{Content DNA Analysis}
To understand the criteria for this filtering, we analyzed the structural characteristics ("Content DNA") of cited versus rejected pages.

\subsection{Tone and Structure}
The data suggests a strong preference for "neutral\_informational" content over "salesy" content. Pages containing structured data, such as comparison tables and pros/cons lists, had a significantly higher probability of being cited than unstructured prose.

\section{Stochasticity and Consistency}
Analysis of cross-run consistency revealed that while core product recommendations remained relatively stable (appearing in 2+ runs), the specific citations used to support these recommendations varied significantly. This "Citation Churn" confirms the stochastic nature of the RAG retrieval process.

\chapter{Discussion}
\label{chap:discussion}

The findings of this study have significant implications for the future of information retrieval, digital marketing, and the architecture of the web itself.

\section{The Convergence of SEO and GEO}
The data supports the hypothesis that Generative Engine Optimization (GEO) is not a replacement for SEO, but its "final form." To be cited by an LLM, a page must first be discoverable (SEO) and then be "extractable" (GEO). The "High-Signal Mandate" suggests that as search costs fall and inference costs remain high, LLMs will increasingly rely on external retrieval, favoring content that is structured, factual, and easily parsed.

\section{The Economic Moat of Retrieval}
We conclude that the future of AI lies not in larger models, but in smarter orchestrators. By using the web as a "distributed memory layer," AI providers can reduce hallucinations and computational costs. However, this creates an economic tension: if AI agents extract answers without sending traffic, the incentive to create high-quality content on the open web may diminish.

\section{The "Invisible Web" for Humans}
Our discovery of the "Visibility Gap" implies that AI agents effectively have access to a "superior" version of the web—one that is not cluttered by ads, pagination loops, or UI suppression. This raises questions about information equity: are human users being served a degraded search experience compared to their AI counterparts?

\section{Implications for Digital Strategy}
For content creators, the message is clear: the era of optimizing for "clicks" is ending. The new era requires optimizing for "citations." This means prioritizing information density, schema markup, and neutral authority over clickbait headlines and aggressive monetization tactics.

\chapter{Conclusion}
\label{chap:conclusion}

This thesis set out to quantify the divergence between traditional search engines and generative AI in the context of commercial product recommendations. By analyzing over 240 search runs and thousands of citations, we have mapped the contours of the "AI Search Filter."

\section{Summary of Findings}
We found that Generative AI does not merely mirror the top search results. Instead, it applies a rigorous, albeit opaque, quality filter that rejects a significant portion of high-ranking content. This filter prioritizes information density, structural clarity, and factual neutrality, often bypassing the "salesy" content that dominates the top of traditional SERPs.

\section{Limitations}
This study was limited to the SaaS and AI software domains and focused primarily on US-centric results. The "Deep Hunt" was capped at Rank 150, and the "Extraction Efficiency" analysis relied on automated scraping which may have missed some dynamic content.

\section{Future Work}
Future research should expand this methodology to other verticals (e.g., health, finance), explore longitudinal trends to see if the "Search Filter" evolves over time, and investigate the impact of multi-modal content (images, video) on grounding behavior.

\section{Final Thoughts}
As we transition from the age of Search to the age of Grounding, the web is being reshaped. The "10 blue links" are fading, replaced by synthesized answers. Understanding the mechanics of this new information economy is no longer optional—it is essential for anyone who wishes to be found in the age of AI.


\appendix

\printglossaries
\printbibliography

\end{document}
